\documentclass{beamer}
\usepackage{amsfonts,amsmath,oldgerm}
\usepackage{ragged2e}
\usepackage{supertabular}
\usepackage{longtable,booktabs,array}
\usepackage{multirow}
\usepackage{calc} % for calculating minipage widths
\usepackage{caption}
\usepackage{xkeyval}

\usetheme{sintef}

\newcommand{\testcolor}[1]{\colorbox{#1}{\textcolor{#1}{test}}~\texttt{#1}}

\usefonttheme[onlymath]{serif}

\titlebackground*{assets/background}

\newcommand{\hrefcol}[2]{\textcolor{cyan}{\href{#1}{#2}}}

\title{Aula 05 - Processos, chamadas de sistema e interrupções}
\subtitle{2023.1 - SOPA2 - Sistemas Operacionais}
\course{Tecnologia em Análise e Desenvolvimento de Sistemas}
\author{\href{mailto:luiz.quirino@ifsp.edu.br}{Luiz \textbf{Quirino}}}
\IDnumber{luiz.quirino@ifsp.edu.br}



\begin{document}
\maketitle

%\begin{frame}
%
%      Este material é produzido utilizando \LaTeX\, baseado na SINTEF Presentation, disponibilizado sob licenciamento \hrefcol{https://creativecommons.org/licenses/by-nc/4.0/legalcode}{Creative Commons CC BY 4.0}
%
%\vspace{\baselineskip}

%In the following you find a brief introduction on how to use \LaTeX\ and the beamer package to prepare slides, based on the one written by \hrefcol{mailto:federico.zenith@sintef.no}{Federico Zenith} for \hrefcol{https://www.overleaf.com/latex/templates/sintef-presentation/jhbhdffczpnx}{SINTEF Presentation}

% This template is released under \hrefcol{https://creativecommons.org/licenses/by-nc/4.0/legalcode}{Creative Commons CC BY 4.0} license

%\end{frame}


\section{Por que precisamos de um sistema operacional?}
\begin{frame}
      

\begin{itemize}
  \item Sistemas de computadores modernos são compostos por diversos dispositivos:
  \item Processadores;
  \item Memória;
  \item Controladoras;
  \item Monitor;
  \item Teclado;
  \item Mouse;
               
                 \item Impressoras;
               
                 \item Etc...
               
               \end{itemize}
                     
     \end{frame}
     
     \begin{frame}
           
     \begin{itemize}
           \item Com tantos dispositivos, surge a necessidade de gerenciamento e manipulação desses diversos dispositivos
         
           \item Tarefa difícil
         
        
     
     
     
     \begin{frame}
           
     \begin{itemize}
           \item Software responsável por gerenciar dispositivos que compõem um sistema computacional e realizar a interação entre o usuário e esses dispositivos;
         
           \item Hardware
         
           \item Processador;
         
           \item Memória Principal;
         
           \item Dispositivos de Entrada/Saída;
         
           \item Software
         
           \item Programas de Aplicação;
         
           \item Programas do Sistema;
         
         \end{itemize}
     \end{frame}


\footlinecolor{}

\backmatter
\end{document}
