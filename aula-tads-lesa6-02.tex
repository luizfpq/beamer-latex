\documentclass{beamer}
\usepackage{amsfonts,amsmath,oldgerm}
\usepackage{ragged2e}

\usetheme{sintef}

\newcommand{\testcolor}[1]{\colorbox{#1}{\textcolor{#1}{test}}~\texttt{#1}}

\usefonttheme[onlymath]{serif}

\titlebackground*{assets/background}

\newcommand{\hrefcol}[2]{\textcolor{cyan}{\href{#1}{#2}}}

\title{Aula 02 - Computação em nuvem PT-2}
\subtitle{2023.1 - LESA 6 -  Laboratório de Escalabilidade de Sistemas}
\course{Tecnologia em Análise e Desenvolvimento de Sistemas}
\author{\href{mailto:luiz.quirino@ifsp.edu.br}{Luiz \textbf{Quirino}}}
\IDnumber{luiz.quirino@ifsp.edu.br}



\begin{document}
\maketitle
\footlinecolor{maincolor}
\section{Estudo de caso}
\begin{frame}
      \frametitle{Estudo de Caso: Expansão da Startup AlphaTech}
      \textbf{Introdução:}
      \begin{itemize}
          \item A AlphaTech é uma startup em crescimento no mercado de educação.
          \item Percebeu a necessidade de migrar seus serviços para a nuvem.
          \item Objetivos: otimizar a escalabilidade, segurança e desempenho.
      \end{itemize}
      \end{frame}
      
      \begin{frame}
      \frametitle{Objetivos}
      \begin{itemize}
          \item Selecionar um provedor de serviços em nuvem.
          \item Dimensionar recursos conforme as necessidades atuais e previstas.
      \end{itemize}
      \end{frame}
      
      \begin{frame}
      \frametitle{Dados Fornecidos}
      \begin{itemize}
          \item Média de 50.000 visitantes únicos/dia. Picos de até 100.000.
          \item Banco de dados de 2 TB, crescendo 10 GB/mês.
          \item Necessidade de resposta rápida para boa experiência do usuário.
          \item Expansão para novos mercados nos próximos dois anos.
      \end{itemize}
      \end{frame}
      
      \begin{frame}
      \frametitle{Tarefas: Pesquisa e Seleção de Provedor}
      \begin{itemize}
          \item Comparação: AWS, Google Cloud e Microsoft Azure.
          \item Criterios: custo, escalabilidade, localização dos data centers, serviços e suporte ao cliente.
      \end{itemize}
      \end{frame}
      
      \begin{frame}
      \frametitle{Tarefas: Dimensionamento de Recursos}
      \begin{itemize}
          \item Definir tipo e tamanho de instância para a aplicação.
          \item Selecionar solução de banco de dados adequada.
          \item Estratégia para picos de tráfego (ex.: auto scaling).
      \end{itemize}
      \end{frame}
      
      \begin{frame}
      \frametitle{Tarefas: Custo Estimado e Planos de Crescimento}
      \begin{itemize}
          \item Estimar custos mensais e anuais.
          \item Considerar descontos por compromissos de longo prazo.
          \item Estratégia para dimensionamento conforme expansão.
      \end{itemize}
      \end{frame}
      
      \begin{frame}
      \frametitle{Entrega e Critérios de Avaliação}
      \textbf{Entrega:}
      \begin{itemize}
          \item Relatório de apresentação para a diretoria.
          \begin{itemize}
            \item Detalhamento de escolha, dimensionamento e custos.
            \item Planos para escalabilidade.
          \end{itemize}
          
      \end{itemize}
      \textbf{Critérios de Avaliação:}
      \begin{itemize}
          \item Decisões lógicas e claras.
          \item Eficiência de custo.
          \item Provisões para crescimento.
          \item Backup, segurança e recuperação de desastres.
      \end{itemize}
      \end{frame}
      

\begin{frame}[fragile]{Imagem do dia}
      \framesubtitle{Computação em nuvem}
	\begin{figure}[H]
		\centerline{\includegraphics[width=0.4\textwidth]{assets/imagem-do-dia/arquivos-na-nuvem-estilingue.jpg}}

	\end{figure}
\end{frame}
\footlinecolor{}

\backmatter
\end{document}
