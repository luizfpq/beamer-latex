\documentclass{beamer}
\usepackage{amsfonts,amsmath,oldgerm}
\usepackage{ragged2e}

\usetheme{sintef}

\newcommand{\testcolor}[1]{\colorbox{#1}{\textcolor{#1}{test}}~\texttt{#1}}

\usefonttheme[onlymath]{serif}

\titlebackground*{assets/background}

\newcommand{\hrefcol}[2]{\textcolor{cyan}{\href{#1}{#2}}}

\title{Aula Zero - Programa da Disciplina}
\subtitle{2023.1 - PDWA5 - Programação Dinâmica para Web}
\course{Tecnologia em Análise e Desenvolvimento de Sistemas}
\author{\href{mailto:luiz.quirino@ifsp.edu.br}{Luiz \textbf{Quirino}}}
\IDnumber{luiz.quirino@ifsp.edu.br}



\begin{document}
\maketitle

%\begin{frame}
%
%      Este material é produzido utilizando \LaTeX\, baseado na SINTEF Presentation, disponibilizado sob licenciamento \hrefcol{https://creativecommons.org/licenses/by-nc/4.0/legalcode}{Creative Commons CC BY 4.0}
%
%\vspace{\baselineskip}

%In the following you find a brief introduction on how to use \LaTeX\ and the beamer package to prepare slides, based on the one written by \hrefcol{mailto:federico.zenith@sintef.no}{Federico Zenith} for \hrefcol{https://www.overleaf.com/latex/templates/sintef-presentation/jhbhdffczpnx}{SINTEF Presentation}

% This template is released under \hrefcol{https://creativecommons.org/licenses/by-nc/4.0/legalcode}{Creative Commons CC BY 4.0} license
%\end{frame}
\footlinecolor{sintefdarkgreen}
\section{Apresentação docente}

\begin{frame}{Sobre o docente:}
Formação:
\begin{itemize}
\item Pós-graduação em Gestão de Riscos e CiberSegurança - Faculdade Focus
\item Sistemas de Informação - UFMS
\item Gestão de Tecnologia da Informação - Unicesumar
\item Técnico em Informática - Centro Paula Souza
\end{itemize}
Áreas de Interesse:
\begin{itemize}
\item Gestão de Tecnologia da Informação
\item Governança de Tecnologia da Informação
\item Gestão de Riscos e CiberSegurança
\item Desenvolvimento de software
\end{itemize}
\end{frame}

\section{Apresentação da disciplina}

\begin{frame}{Ementa}\justifying
      A disciplina exercita conceitos de desenvolvimento de aplicações web, considerando a
      utilização de linguagens dinâmicas e o desenvolvimento de Software como Serviço (SaaS).
\end{frame}

\begin{frame}{Objetivo da disciplina}\justifying
      Desenvolver aplicações web utilizando linguagens dinâmicas. Aplicar técnicas de desenvolvimento para a construção de software como serviço. Conhecer as diferentes abordagens neste contexto e as suas vantagens e desvantagens.
\end{frame}

\begin{frame}{Conteúdo Programático}\justifying
      \begin{itemize}
            \item Software como Serviço e o desenvolvimento de aplicações Web
            \item Tipos de aplicação web: Single Page Applications, Progressive Web Applications e o
            modelo clássico;
            \item Fundamentos de linguagens dinâmicas, suas vantagens e desvantagens;
            \item Construção de serviços e API’s
                  \begin{itemize}
                        \item REST / SOAP
                        \item Documentação e teste de API (Swaggers)
                        \item Outras abordagens (CQRS, GraphQL)
                  \end{itemize}
            
            
      \end{itemize}
\end{frame}

\begin{frame}{Conteúdo Programático}\justifying
      \begin{itemize}
            \item Ajax e comunicação assíncrona
            \item Desenvolvimento de páginas dinâmicas
            \item Processamento de requisições
            \item Arquitetura de aplicações Web
            \item Persistência de dados
            \item Armazenamento e recuperação de arquivos
           
      \end{itemize}
\end{frame}

\begin{frame}{Conteúdo Programático}\justifying
      \begin{itemize}
            
            \item Utilização de computação em nuvem
            \item Segurança em aplicações Web
            \item Internacionalização
            \item Autenticação e Autorização
            \item Comunicação stateful (Websockets)
      \end{itemize}
\end{frame}


\begin{frame}{Bibliografia básica}\justifying
      \begin{itemize}
            \item \textbf{FLANAGAN, David. \textcolor{sintefdarkgreen}{Javascript: o guia definitivo. 6. ed.}} São Paulo: Bookman, 2012. ISBN 9788565837194.\\
            \item \textbf{FOX, Armando; PATTERSON, David. \textcolor{sintefdarkgreen}{Desenvolvimento de software como serviço (SaaS): uma abordagem ágil usando computação em nuvem.}} Strawberry Canyon LLC, 2015. ASIN B010C83AOC.\\
            \item \textbf{RICHARDSON, Leonard. \textcolor{sintefdarkgreen}{RESTful Web APIs.}} Califórnia: O’Reilly, 2014. ISBN 9781449358068. \\
      \end{itemize}
\end{frame}
      

\begin{frame}{Bibliografia complementar}
      \begin{itemize}
            \item \textbf{BERNSTEIN, T. et al. \textcolor{sintefdarkgreen}{Segurança na Internet.}}  Rio de Janeiro: Campus, 1997. ISBN
            8535201408.
            \item \textbf{ÇELIK, Tantek et al. \textcolor{sintefdarkgreen}{Cascading stylesheets level 2 revision 2 (CSS 2.2) specification.}} W3C Consortium, 2016. Disponível em: https://www.w3.org/TR/CSS22/. Acesso em: 14 jun. 2019.
            \item \textbf{GREENBERG, J.; LAKELAND, J. R. \textcolor{sintefdarkgreen}{A methodology for developing and deploying: Internet \& Intranet solutions.}} Michigan: H P. Prentice Hall. 1998. ISBN 9780132096775.
            
           
      \end{itemize}
\end{frame}

\begin{frame}{Bibliografia complementar}
      \begin{itemize}
            \item \textbf{SOARES, Walace.} \textbf{\textcolor{sintefdarkgreen}{Programação WEB com PHP 5. 1. ed.}} São Paulo: Erica, 2014. ISBN 9788536507729.
            \item \textbf{UMAR, Amjad. \textcolor{sintefdarkgreen}{Object-Oriented Client/Server Internet environments. 1. ed.}} Upper Saddle River: Prentice Hall, 1997. ISBN 9780133755442.
            \item \textbf{WELLING, Luke; THOMSON, Laura. \textcolor{sintefdarkgreen}{Php e Mysql: desenvolvimento web.}} Rio de Janeiro: Elsevier, 2005. ISBN 9788535217148.

      \end{itemize}
\end{frame}

\section{Planejamento}


\begin{frame}[fragile]{Planejamento de aulas}
      \begin{columns}
            \begin{column}{0.5\textwidth}
                  \textbf{Definições da disciplina:}
                  \begin{itemize}
                        \item Total de aulas: 57 
                        \item Aulas semanais: 3 (19 semanas)
                        \item Total de horas: 42,75
                        \item Atividades práticas (divididas em entregas mensais, a partir de setembro)
                        \item Atividades teoricas (discussões em sala)
                        \item Desenvolvimento de projeto final
      
                  \end{itemize}


            \end{column}
            \begin{column}{0.5\textwidth}
                 \textbf{Turma: 324019}
                  \begin{itemize}
                        \item Quintas-feiras: 18:50 - 21:05
                        \item Período: 27/07 - 30/11 (07/12 - IFA)
                        \item Feriados: 07/09 - 12/10 - 02/11
                        \item Eventos: 21/09
                  \end{itemize}
                  \textbf{Turma: 324060}
                  \begin{itemize}
                        \item Sextas-feiras: 18:50 - 21:05
                        \item Período: 28/07 - 01/12 (08/12 - IFA)
                        \item Feriados: 08/09 - 13/10 - 03/11
                        \item Eventos: 22/09
                  \end{itemize}
            \end{column}
      \end{columns}
\end{frame}



\section{Moodle}

\begin{frame}[fragile]{Auto Inscrição Moodle - Turma Quinta-feira}

      \textbf{Autoinscrição → \textcolor{sintefred}{Chave: IFSP@2023.2}}

      \begin{figure}[H]
            \centerline{\includegraphics[width=1\textwidth]{assets/aula-tads-pdwa5/moodle_quinta.png}}
            
        \end{figure}
        
\end{frame}

\begin{frame}[fragile]{Auto Inscrição Moodle - Turma Sexta-feira}

      \textbf{Autoinscrição → \textcolor{sintefred}{Chave: IFSP@2023.2}}
        \begin{figure}[H]
            \centerline{\includegraphics[width=1\textwidth]{assets/aula-tads-pdwa5/moodle_sexta.png}}
            
        \end{figure}
\end{frame}

\section{Avaliações}

\begin{frame}[fragile]\justifying
\frametitle{Sistema de avaliação}
\begin{itemize}
            
            \item Como seremos avaliados:
            \begin{itemize}
                  \item Um trabalho teórico(TT) que atenderá 25\% da nota;
                  \item Um trabalho prático(TP) que atenderá 45\% da nota;
                  \item Atividades avaliativas/participação em aula (AA), contando como 30\% da nota;
            \end{itemize}
            \item Em caso de não obtenção dos critérios mínimos para aprovação, aplicação de IFA por meio de prova teórica, escrita, presencial;
\end{itemize}
\begin{colorblock}[black]{sinteflightgreen}{ATENÇÃO}
      Os trabalhos serão disponibilizados a \textbf{partir da 5ª aula ministrada} na disciplina, contando com documentação disponibilizado pelo docente, ficando aberta para entregas sucessivas no moodle.
      Os trabalhos teórico e prático são interdepentes, sendo a parte teórica / descritiva fudamentada na implementação da atividade prática;
\end{colorblock}

\end{frame}

\begin{frame}[fragile]\justifying
      \frametitle{Sistema de avaliação}
      \begin{itemize}
            \item Média trabalho \[ MT = TP * 0,45 + TT * 0,25\]
            \item Média produtividade \[ MP = \left ( \frac{AA_1 + AA_2 + ... + AA_n}n \right ) * 0,3 \]
            \item Média Final - MF \[MF = MT + MP\]
      \end{itemize}
      
      \end{frame}


\begin{frame}[fragile]\justifying
      \frametitle{Critérios de avaliação}
      Respeitando ao disposto no PPC vigente do curso, no item \textbf{\textit{8. AVALIAÇÃO DA APRENDIZAGEM. }}
      \newline
      \newline
      \textit{Os critérios de aprovação nos componentes curriculares, envolvendo simultaneamente frequência e avaliação, para os cursos da Educação Superior de 
      regime semestral, são a obtenção, no componente curricular, de nota semestral igual ou superior a 6,0 (seis) e frequência mínima de 75\% (setenta e cinco por cento) das
      aulas e demais atividades. }
\end{frame}

\begin{frame}[fragile]\justifying
      \frametitle{Critérios de avaliação}
      \textit{Fica sujeito ao Instrumento Final de Avaliação (IFA), o estudante que obtenha, no componente curricular, nota semestral igual ou superior a
      4,0 (quatro) e inferior a 6,0 (seis) e frequência mínima de 75\% (setenta e cinco por cento) das aulas e demais atividades. O estudante que realizar o Instrumento Final de
      Avaliação, para ser aprovado, deverá obter a nota mínima igual a 6,0 (seis). A nota final considerada, para registros escolares, será a maior entre a nota semestral e a nota do
      Instrumento Final de Avaliação (IFA). 
      \newline
      \newline
      É importante ressaltar que os critérios de avaliação na Educação Superior primam pela autonomia intelectual.}
\end{frame}

\section{Conduta ética}
\begin{frame}
\frametitle{Termos de Conduta}
      \begin{itemize}
            \item Trabalhos e provas devem ser feitas INDIVIDUALMENTE;
            \item Cada estudante tem responsabilidade sobre cópias de suas implementações e provas, mesmo que parciais;
            \item Não faça implementeções em grupo e não compartilhe programas ou trechos de programas;
            \item Você pode consultar seus colegas para esclarecer dúvidas e discutir idéias sobre implementações, mas NÃO copie programas!
            \item Implementações e provas consideradas plagiadas terão nota ZERO;
            \item O estudante que se envolver em DOIS CASOS DE PLÁGIO estará automaticamente REPROVADO na disciplina.
      \end{itemize}
\end{frame}

\section{Informações sobre os slides}

\footlinecolor{sintefyellow}
\begin{frame}
      
      \begin{itemize}
            \item Slides com rodapé em vermelho foram adicionados após a aula dada;
            \item Slides com rodapé em amarelo foram atualizados  após a aula dada.
      \end{itemize}
\end{frame}

\footlinecolor{}

\backmatter
\end{document}
