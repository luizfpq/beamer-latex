\documentclass{beamer}
\usepackage{amsfonts,amsmath,oldgerm}
\usepackage{ragged2e}

\usetheme{sintef}

\newcommand{\testcolor}[1]{\colorbox{#1}{\textcolor{#1}{test}}~\texttt{#1}}

\usefonttheme[onlymath]{serif}

\titlebackground*{assets/background}

\newcommand{\hrefcol}[2]{\textcolor{cyan}{\href{#1}{#2}}}

\title{Planejamento volta as aulas}
\subtitle{2023.1 - SPOIFDS - Informática e Ferramentas para Desenvolvimento }
\course{TÉC. DES. DE SISTEMAS INTEGRADO}
\author{\href{mailto:luizfpq@gmail.com}{Luiz \textbf{Quirino}}}
\IDnumber{luizfpq@gmail.com}



\begin{document}
\maketitle

%\begin{frame}
%
%      Este material é produzido utilizando \LaTeX\, baseado na SINTEF Presentation, disponibilizado sob licenciamento \hrefcol{https://creativecommons.org/licenses/by-nc/4.0/legalcode}{Creative Commons CC BY 4.0}
%
%\vspace{\baselineskip}

%In the following you find a brief introduction on how to use \LaTeX\ and the beamer package to prepare slides, based on the one written by \hrefcol{mailto:federico.zenith@sintef.no}{Federico Zenith} for \hrefcol{https://www.overleaf.com/latex/templates/sintef-presentation/jhbhdffczpnx}{SINTEF Presentation}

% This template is released under \hrefcol{https://creativecommons.org/licenses/by-nc/4.0/legalcode}{Creative Commons CC BY 4.0} license

%\end{frame}


\section{Planejamento}


\begin{frame}[fragile]{Planejamento de aulas}
      \begin{columns}
            \begin{column}{0.5\textwidth}
                  \textbf{Definições da disciplina:}
                  \begin{itemize}
                        \item Total de aulas: 57 
                        \item Aulas semanais: 2 (17 semanas restantes)
                        \item Total de horas: 57
                        \item Atividades práticas (divididas em entregas mensais, a partir de setembro)
                        \item Atividades teoricas (discussões em sala)
                        \item Desenvolvimento de projeto final
      
                  \end{itemize}


            \end{column}
            \begin{column}{0.6\textwidth}
                  \begin{column}{0.6\textwidth}
                        \textbf{Turmas A e B: }
                        \begin{itemize}
                              \item Terças-feiras: 13:15 - 16:30
                              \item Período: 01/08 - 21/11 \\ (28/11 - IFA)
                        \end{itemize}
                        
                  \end{column}
                  
            \end{column}
      \end{columns}
\end{frame}


\section{Conteúdo programático}

\begin{frame}
      \begin{itemize}
            \item Versionamento
            \item Terminal e comandos
            \item Qualidade de código
            \item Ferramentas de escritório
            \item LaTeX
      \end{itemize}
\end{frame}


\section{Avaliações}

\begin{frame}[fragile]\justifying
\frametitle{Sistema de avaliação}
\begin{itemize}
            
            \item Como seremos avaliados:
            \begin{itemize}
                  \item Um trabalho teórico(TT) que atenderá 40\% da nota;
                  \item Um trabalho prático(TP) que atenderá 40\% da nota;
                  \item Atividades avaliativas/participação em aula (AA), contando como 20\% da nota;
            \end{itemize}
            \item Em caso de não obtenção dos critérios mínimos para aprovação, aplicação de IFA por meio de prova teórica, escrita, presencial;
\end{itemize}
\begin{colorblock}[black]{sinteflightgreen}{ATENÇÃO}
      Os trabalhos serão disponibilizados a \textbf{partir de 29/08}, contando com documentação disponibilizado pelo docente, ficando aberta para entregas sucessivas no moodle.
      Os trabalhos teórico e prático são interdepentes, sendo a parte teórica / descritiva fudamentada na implementação da atividade prática;
\end{colorblock}

\end{frame}


\begin{frame}[fragile]\justifying
      \frametitle{Sistema de avaliação}
      \begin{itemize}
            \item Média trabalho \[ MT = TP * 0,40 + TT * 0,40\]
            \item Média produtividade \[ MP = \left ( \frac{AA_1 + AA_2 + ... + AA_n}n \right ) * 0,2 \]
            \item Média Final - MF \[MF = MT + MP\]
      \end{itemize}
      
      \end{frame}
      \footlinecolor{}

\begin{frame}[fragile]\justifying
      \frametitle{Critérios de avaliação}
      Respeitando ao disposto no PPC vigente do curso, no item \textbf{\textit{8. AVALIAÇÃO DA APRENDIZAGEM. }}
      \newline
      \newline
      \textit{Os critérios de aprovação nos componentes curriculares, envolvendo simultaneamente frequência e avaliação, para os cursos da Educação Superior de 
      regime semestral, são a obtenção, no componente curricular, de nota semestral igual ou superior a 6,0 (seis) e frequência mínima de 75\% (setenta e cinco por cento) das
      aulas e demais atividades. }
\end{frame}

\begin{frame}[fragile]\justifying
      \frametitle{Critérios de avaliação}
      \textit{Fica sujeito ao Instrumento Final de Avaliação (IFA), o estudante que obtenha, no componente curricular, nota semestral igual ou superior a
      4,0 (quatro) e inferior a 6,0 (seis) e frequência mínima de 75\% (setenta e cinco por cento) das aulas e demais atividades. \\ O estudante que realizar o Instrumento Final de
      Avaliação, para ser aprovado, deverá obter a nota mínima igual a 6,0 (seis). A nota final considerada, para registros escolares, será a maior entre a nota semestral e a nota do
      Instrumento Final de Avaliação (IFA). }
      \newline
\end{frame}

\section{Conduta ética}
\begin{frame}
\frametitle{Termos de Conduta}
      \begin{itemize}
            \item Trabalhos e provas devem ser feitas INDIVIDUALMENTE;
            \item Cada estudante tem responsabilidade sobre cópias de suas implementações e provas, mesmo que parciais;
            \item Não faça implementeções em grupo e não compartilhe programas ou trechos de programas;
            \item Você pode consultar seus colegas para esclarecer dúvidas e discutir idéias sobre implementações, mas NÃO copie programas!
            \item Implementações e provas consideradas plagiadas terão nota ZERO;
            \item O estudante que se envolver em DOIS CASOS DE PLÁGIO estará automaticamente REPROVADO na disciplina.
      \end{itemize}
\end{frame}

\section{Informações sobre os slides}

\footlinecolor{sintefyellow}
\begin{frame}[fragile]{Dicas sobre os slides}
      
      \begin{itemize}
            \item Slides com rodapé em vermelho foram adicionados após a aula dada;
            \item Slides com rodapé em amarelo foram atualizados  após a aula dada.
      \end{itemize}
\end{frame}

\footlinecolor{sintefred}
\begin{frame}[fragile]{Imagem do dia}

        \begin{figure}[H]
            \centerline{\includegraphics[width=0.4\textwidth]{assets/imagem-do-dia/ferias_acabou.jpg}}
            
        \end{figure}
\end{frame}


\footlinecolor{}



\backmatter
\end{document}
