\documentclass{beamer}
\usepackage{amsfonts,amsmath,oldgerm}
\usepackage{ragged2e}

\usetheme{sintef}

\newcommand{\testcolor}[1]{\colorbox{#1}{\textcolor{#1}{test}}~\texttt{#1}}

\usefonttheme[onlymath]{serif}

\titlebackground*{assets/background}

\newcommand{\hrefcol}[2]{\textcolor{cyan}{\href{#1}{#2}}}

\title{Aula 01 - História dos sistemas operacionais}
\subtitle{2023.1 - SPOSOPE - Sistemas Operacionais}
\course{Tecnologia em Análise e Desenvolvimento de Sistemas}
\author{\href{mailto:luizfpq@gmail.com}{Luiz \textbf{Quirino}}}
\IDnumber{luizfpq@gmail.com}



\begin{document}
\maketitle

%\begin{frame}
%
%      Este material é produzido utilizando \LaTeX\, baseado na SINTEF Presentation, disponibilizado sob licenciamento \hrefcol{https://creativecommons.org/licenses/by-nc/4.0/legalcode}{Creative Commons CC BY 4.0}
%
%\vspace{\baselineskip}

%In the following you find a brief introduction on how to use \LaTeX\ and the beamer package to prepare slides, based on the one written by \hrefcol{mailto:federico.zenith@sintef.no}{Federico Zenith} for \hrefcol{https://www.overleaf.com/latex/templates/sintef-presentation/jhbhdffczpnx}{SINTEF Presentation}

% This template is released under \hrefcol{https://creativecommons.org/licenses/by-nc/4.0/legalcode}{Creative Commons CC BY 4.0} license

%\end{frame}

\section{História dos Sistemas Operacionais}

\begin{frame}{Introdução}
      \begin{itemize}
            \item Os sistemas operacionais são fundamentais para o funcionamento dos computadores modernos.
            \item Eles evoluíram ao longo do tempo para fornecer recursos avançados e eficientes.
            \item Nesta aula, exploraremos brevemente a história dos sistemas operacionais.
          \end{itemize}

\end{frame}


\section{Primeira geração}
\begin{frame}{Primeira geração: 1945 a 1955}
      \begin{itemize}
          \item Computadores: em meados da década de 40
          \item Máquinas de Calcular
          \item Relés mecânicos
          \item Ciclos medidos em segundos (!)
          \item Posteriormente os relés foram substituídos por válvulas a vácuo
          \item Máquinas enormes que ocupavam uma sala inteira
      \end{itemize}
  \end{frame}

  \begin{frame}{Primeira Geração - 1945 a 1955}
      \begin{itemize}
          \item Um único grupo de pessoas projetava, construía, programava, operava e mantinha cada máquina
          \item Toda a programação feita em linguagem de máquina pura (!)
          \item Não existia nenhuma linguagem de programação, nem mesmo Assembly!
      \end{itemize}
  \end{frame}
  \begin{frame}{Primeira Geração - 1945 a 1955}
      \begin{itemize}
          \item Não existia o termo sistemas operacional
          \item O programador reservava um período de tempo
          \item Inseria seu painel de conectores no computador
          \item Cálculos numéricos simples, como geração de tabelas de senos, co-senos e logaritmos
          \item No início dos anos 50 as coisas “melhoraram” com os cartões perfurados
      \end{itemize}
  \end{frame}
  
            
  \section{Segunda geração}
  \begin{frame}{Segunda Geração - 1955 a 1965}
      \begin{itemize}
          \item A introdução do transistor ocorreu em meados da década de 50
          \item Os computadores se tornaram confiáveis para serem vendidos para clientes
          \item Houve uma separação clara entre projetistas, construtores, operadores, programadores e pessoal de manutenção
          \item Chamados de Grande Porte ou Mainframes
      \end{itemize}
  \end{frame}
  \begin{frame}{Segunda Geração - 1955 a 1965}
      \begin{itemize}
          \item Eram postas em salas especiais com ar-condicionado e operadores profissionais
          \item Nas mãos de grandes empresas, órgãos do governo, universidades
          \item Processo de execução de um job (tarefa):
                programava em Fortran/Assembly no papel,
                convertia para cartão perfurado, entregava para um operador, resultado na sala de saída
      \end{itemize}
  \end{frame}
  \begin{frame}{Segunda Geração - 1955 a 1965}
      \begin{itemize}
          \item Se o compilador Fortran fosse necessário, o seu código (cartões) precisavam ser inseridos
          \item Existia o tempo ocioso (sala de máquina, submissão e saída)
          \item Solução: sistema de processamento em lotes (batch) usando um computador bom para isso (ex: IBM 1401) e a fita iria para o computador de cálculos (IBM 7094)
      \end{itemize}
  \end{frame}
  \begin{frame}{Segunda Geração - 1955 a 1965 (Fluxo do Processamento em Batch)}
      \begin{itemize}
          \item Após a geração da fita de jobs pelo computador mais simples...
          \item ...o operador carregava um programa especial, que lia o primeiro programa na fila e o executava, gravando a saída em outra fita
          \item Isso era feito até acabarem os jobs
          \item Esse programa especial é o ancestral dos sistemas operacionais
      \end{itemize}
  \end{frame}
  \begin{frame}{Segunda Geração - 1955 a 1965 (Fluxo do Processamento em Batch)}
      \begin{itemize}
          \item Eram usados principalmente para cálculos de engenharia e científicos
          \item Eram escritos em Fortran e Assembly
          \item Sistemas Operacionais:
                \begin{itemize}
                    \item FMS = Fortran Monitor System
                    \item IBSYS = Sistema Operacional do IBM 7094
                \end{itemize}
      \end{itemize}
  \end{frame}
  
  \section{Terceira geração}
  \begin{frame}{Terceira Geração - 1965 a 1980}
      \begin{itemize}
          \item No início da década de 60, os fabricantes tinham duas linhas de produtos incompatíveis:
                \begin{itemize}
                    \item Mainframes (7094): engenharia/ciência
                    \item Comerciais (1401): bancos/seguradoras
                \end{itemize}
          \item Manter as duas linhas era muito caro
          \item Além disso, alguns clientes começavam com máquinas pequenas e migravam para maiores
      \end{itemize}
  \end{frame}
  \begin{frame}{Terceira Geração - 1965 a 1980}
      \begin{itemize}
          \item IBM tentou resolver isso criando System/360
          \item Era uma série de máquinas de software compatível com o 1401 e o 7094
          \item A diferença das máquinas de 360 era no preço e no desempenho, pois variavam o hardware:
                \begin{itemize}
                    \item Capacidade máxima de memória, velocidade do processador, número de dispositivos de E/S
                \end{itemize}
          \item Mas as máquinas tinham a mesma arquitetura!
      \end{itemize}
  \end{frame}
  \begin{frame}{Terceira Geração - 1965 a 1980}
      \begin{itemize}
          \item Mesma arquitetura = Mesmo ISA! (Instruction Set Architecture)
          \item O 360 foi projetado para computação científica (numérica) e comercial
          \item Primeira geração importante a utilizar CIs (Circuitos Integrados)
          \item Assim, uma única família de máquinas poderia satisfazer as necessidades de todos os clientes
          \item Nos anos seguintes, a IBM lançou computadores sucessores: 370, 4300, 3080, 3090 e Z
      \end{itemize}
  \end{frame}
  \begin{frame}{Terceira Geração - 1965 a 1980}
      \begin{itemize}
          \item A intenção era que todo software (incluindo o SO OS/360) funcionasse em todos os modelos
          \item Para uso científico e comercial
          \item Ser eficiente em ambientes muito distintos
          \item Não deu certo...
          \item O resultado foi um SO enorme e complexo
          \item Milhões de linhas em Assembly, milhares de programadores e milhares de erros
      \end{itemize}
  \end{frame}
  \begin{frame}{Terceira Geração - Multiprogramação}
      \begin{itemize}
          \item Apesar do tamanho enorme e de seus erros, os SOs de 3ª geração atendiam aos clientes
          \item Popularizaram técnicas importantes, como o conceito de multiprogramação
      \end{itemize}
  \end{frame}
  \begin{frame}{Terceira Geração - Spooling}
      \begin{itemize}
          \item Vários jobs simultâneos precisam de ajuda do hardware (segurança, memória, etc)
          \item O 360 e outros hardwares de 3ª geração estavam preparados para isso
          \item Outro conceito importante é o Spooling (Operação Periférica Simultânea On-line)
          \item Efetuava a leitura de jobs de cartões para o disco – eliminava a função dos IBM 1401 (ler os cartões separadamente)
      \end{itemize}
  \end{frame}
  \begin{frame}{Terceira Geração - Time Sharing}
      \begin{itemize}
          \item Ainda eram sistemas em lote, pois os jobs continuavam executando em sequência
          \item Abriu espaço para o compartilhamento de tempo (time sharing)
          \item Na verdade, esse conceito é uma variante de multiprogramação
          \item Gerenciar o tempo ocioso entre as aplicações
      \end{itemize}
  \end{frame}
  \begin{frame}{Terceira Geração - Time Sharing e CTSS}
      \begin{itemize}
          \item CTSS: Primeiro SO sério com time sharing
          \item CTSS = Compatible Time Sharing System
          \item Desenvolvido pelo MIT em um IBM 7094 por volta de 1962
          \item O conceito de time sharing somente tornou-se popular quando o hardware passou a dar suporte a ele
      \end{itemize}
  \end{frame}
  \begin{frame}{Terceira Geração - MULTICS}
      \begin{itemize}
          \item CTSS, Bell Labs e G.E. (General Electric)
          \item Decidiram dar o início a um computer utility
          \item Máquina que comportaria centenas de usuários de tempo compartilhado
          \item A mesma lógica da rede elétrica (você pluga na tomada e utiliza)
          \item MULTICS = Serviço de Computação e Informação Multiplexado
      \end{itemize}
  \end{frame}
  \begin{frame}{Terceira Geração - MULTICS}
      \begin{itemize}
          \item A ideia do MULTICS era uma super máquina oferecendo poder computacional para todos na região de Boston
          \item A máquina não tomou conta do mundo, porém muitas ideias surgiram ali
          \item O GE e a Bell Labs saíram e o MIT continuou
          \item Foi utilizado pela GE, Ford e Agência Nacional de Segurança dos EUA até o final dos anos 90
          \item Deu origem aos conceitos de LANs (Redes Locais) e Sistemas Distribuídos
      \end{itemize}
  \end{frame}
  \begin{frame}{Terceira Geração - UNIX}
      \begin{itemize}
          \item PDP-1 da DEC, com 4K de palavras de 18 bits
          \item Custava \$120.000, menos de 5\% do valor do IBM 7094
          \item Não era para a parte numérica e deu origem a uma sequência de PDFs incompatíveis
          \item Ken Thompson trabalhou no MULTICS e, usando um PDP-7, criou uma versão monousuária do mesmo (o futuro UNIX)
      \end{itemize}
  \end{frame}
  \begin{frame}{Terceira Geração - POSIX}
      \begin{itemize}
          \item Source disponível deu origem a várias versões incompatíveis, gerando um caos
          \item Duas versões importantes:
                \begin{itemize}
                    \item System V (AT\&T)
                    \item BSD (Berkeley Software Distribution)
                          \begin{itemize}
                              \item FreeBSD, OpenBSD e NetBSD
                          \end{itemize}
                \end{itemize}
          \item Para poder escrever programas que rodassem em qualquer UNIX, o IEEE desenvolveu o POSIX
      \end{itemize}
  \end{frame}
  
  \begin{frame}{Terceira Geração - POSIX}
      \begin{itemize}
          \item Define uma interface mínima de chamadas de sistemas que os sistemas UNIX devem suportar
          \item Outros sistemas operacionais que não são baseados em UNIX também adotaram o padrão
      \end{itemize}
  \end{frame}
  
  
\section{Quarta geração - era moderna}
\begin{frame}{Quarta Geração - 1980 - hoje (Era moderna)}
      \begin{itemize}
          \item Circuitos LSI (Integração em Larga Escala)
          \item Chips com milhares de transistores em um centímetro quadrado de silício
          \item PCs baseados em microprocessadores
          \item Inspirados nos minicomputadores (PDP-11), porém muito mais baratos
          \item Qualquer pessoa poderia ter um computador
      \end{itemize}
  \end{frame}
  \begin{frame}{Quarta Geração - 1980 - hoje}
      \begin{itemize}
          \item Intel 8080 (1974): primeiro processador de 8 bits
          \item Z80: compatível com o 8080, feito pela Zilog
          \item SO CP/M (Control Program for Microcomputers)
              \begin{itemize}
                  \item Criado pela Digital Research
                  \item Dominou o mercado de PCs por 5 anos
              \end{itemize}
          \item Motorola 6800 de 8 bits
          \item MOS Technology 6502 (usado no Apple II)
      \end{itemize}
  \end{frame}
  \begin{frame}{Quarta Geração - Microsoft}
      \begin{itemize}
          \item As placas do CP/M eram comercializadas por uma pequena empresa chamada Microsoft
          \item A Microsoft tinha um nicho de mercado: interpretador BASIC para o CP/M
          \item Geração seguinte: processadores de 16 bits
          \item Intel 8086 e o Intel 8088
          \item A IBM projetou o IBM PC com o 8088
      \end{itemize}
  \end{frame}
  \begin{frame}{Quarta Geração - Microsoft}
      \begin{itemize}
          \item Microsoft ofereceu um sistema para a IBM que incluía o BASIC e um SO: o DOS
          \item DOS = Disk Operating System
          \item Foi um SO adaptado de outro (comprado) que se tornou o MS-DOS
          \item Sistemas de linha de comandos:
                \begin{itemize}
                    \item CP/M
                    \item DOS
                    \item Apple DOS
                \end{itemize}
      \end{itemize}
  \end{frame}
  \begin{frame}{Quarta Geração - GUI e Apple}
      \begin{itemize}
          \item Dough Engelbart (Stanford Research Institute) criou a GUI (Graphical User Interface)
          \item A Apple “incorporou” essa ideia ao Macintosh
                \begin{itemize}
                    \item Foi lançado em 1984
                    \item CPU Motorola 6800 de 16 bits
                    \item 64Kb de ROM para a GUI
                \end{itemize}
          \item Apple mudou para o PowerPC de 32 bits
          \item Em 2001 lançou o Mac OS X (Unix based)
          \item Em 2005 migrou para processadores Intel
      \end{itemize}
  \end{frame}
  \begin{frame}{Quarta Geração - Windows}
      \begin{itemize}
          \item Originalmente um ambiente gráfico rodando sobre o DOS em 16 bits
          \item Os descendentes do Windows NT são sistemas gráficos completos de 32 ou 64 bits
      \end{itemize}
  \end{frame}
  \begin{frame}{Quarta Geração - UNIX e Linux}
      \begin{itemize}
          \item Poderoso em estações de trabalho e servidores (especialmente processadores RISC)
          \item Computadores baseados no Pentium (Celeron, Xeon, AMD, etc) adotaram o Linux como uma alternativa ao Windows
          \item X Window = sistema de janelas desenvolvido pelo M.I.T.
          \item Serve de base para uma GUI: KDE, Gnome, etc
      \end{itemize}
  \end{frame}
  \begin{frame}{Quarta Geração - Rede e Distribuídos}
      \begin{itemize}
          \item Meados dos anos 80
          \item Serviços compartilhados
          \item Servidores remotos
          \item Middleware (Transparência)
          \item Reconhecer múltiplos processadores / núcleos
          \item Deu origem à área de Sistemas Distribuídos
      \end{itemize}
  \end{frame}
  





\section{EXEMPLO de Proposta de atividade}

\begin{frame}{Explorando a Evolução dos Sistemas Operacionais}

      Objetivo:
      \begin{itemize}
            \item  Investigar e compreender a evolução dos sistemas operacionais ao longo do tempo, desde os sistemas de lote até os modernos sistemas com interfaces gráficas.
      \end{itemize}
      
\end{frame}

\begin{frame}{Referências}\justifying
      \begin{itemize}
            \item \textbf{Slides SOPA2 - Prof. Alexandre Beletti Ferreira}
            \item \textbf{TANENBAUM, Andrew S.; WOODHULL, Albert S.} Sistemas operacionais: projeto e implementação. 3. ed. Porto Alegre: Bookman, 2008. ISBN 9788577800571.
      \end{itemize}
\end{frame}

\begin{frame}[fragile]{Imagem do dia}

    \begin{figure}[H]
        \centerline{\includegraphics[width=0.8\textwidth]{assets/imagem-do-dia/primeiro-dia-aula.jpg}}
        
    \end{figure}
\end{frame}

\backmatter
\end{document}
