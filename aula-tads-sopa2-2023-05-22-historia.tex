\documentclass{beamer}
\usepackage{amsfonts,amsmath,oldgerm}
\usepackage{ragged2e}

\usetheme{sintef}

\newcommand{\testcolor}[1]{\colorbox{#1}{\textcolor{#1}{test}}~\texttt{#1}}

\usefonttheme[onlymath]{serif}

\titlebackground*{assets/background}

\newcommand{\hrefcol}[2]{\textcolor{cyan}{\href{#1}{#2}}}

\title{Aula 01 - História dos sistemas operacionais}
\subtitle{2023.1 - SOPA2 - Sistemas Operacionais}
\course{Tecnologia em Análise e Desenvolvimento de Sistemas}
\author{\href{mailto:luizfpq@gmail.com}{Luiz \textbf{Quirino}}}
\IDnumber{luizfpq@gmail.com}



\begin{document}
\maketitle

%\begin{frame}
%
%      Este material é produzido utilizando \LaTeX\, baseado na SINTEF Presentation, disponibilizado sob licenciamento \hrefcol{https://creativecommons.org/licenses/by-nc/4.0/legalcode}{Creative Commons CC BY 4.0}
%
%\vspace{\baselineskip}

%In the following you find a brief introduction on how to use \LaTeX\ and the beamer package to prepare slides, based on the one written by \hrefcol{mailto:federico.zenith@sintef.no}{Federico Zenith} for \hrefcol{https://www.overleaf.com/latex/templates/sintef-presentation/jhbhdffczpnx}{SINTEF Presentation}

% This template is released under \hrefcol{https://creativecommons.org/licenses/by-nc/4.0/legalcode}{Creative Commons CC BY 4.0} license

%\end{frame}

\section{História dos Sistemas Operacionais}

\begin{frame}{Introdução}
      \begin{itemize}
            \item Os sistemas operacionais são fundamentais para o funcionamento dos computadores modernos.
            \item Eles evoluíram ao longo do tempo para fornecer recursos avançados e eficientes.
            \item Nesta aula, exploraremos brevemente a história dos sistemas operacionais.
          \end{itemize}

\end{frame}

\begin{frame}{Primeira geração}
      \begin{itemize}
            \item Na década de 1950, os sistemas operacionais de lote foram desenvolvidos.
            \item Os programas eram submetidos em lote para execução em lotes, sem interação direta do usuário.
            \item Exemplos de sistemas operacionais dessa época incluem IBM OS/360 e OS/MFT.
          \end{itemize}
\end{frame}
            
      
\begin{frame}{Segunda geração}\justifying
      \begin{itemize}
            \item Nos anos 1960, surgiram os sistemas operacionais de tempo compartilhado.
            \item Vários usuários podiam interagir com o computador simultaneamente.
            \item Exemplos de sistemas operacionais dessa época incluem CTSS e MULTICS.
          \end{itemize}
\end{frame}

\begin{frame}{Terceira geração}
      \begin{itemize}
            \item Na década de 1970, os sistemas operacionais com multiprogramação se tornaram populares.
            \item Vários programas podiam ser executados simultaneamente.
            \item Exemplos de sistemas operacionais dessa época incluem UNIX e VMS.
          \end{itemize}
\end{frame}

\begin{frame}{Era moderna}

      \begin{itemize}
            \item A partir dos anos 1980, os sistemas operacionais evoluíram para incluir interfaces gráficas de usuário (GUI).
            \item Surgiram sistemas operacionais populares, como o Windows e o Mac OS.
            \item Hoje, temos uma ampla variedade de sistemas operacionais, incluindo Linux, Windows, macOS e Android.
          \end{itemize}
      \end{frame}

\section{Proposta de atividade}

\begin{frame}{Explorando a Evolução dos Sistemas Operacionais}

      Objetivo:
      \begin{itemize}
            \item  Investigar e compreender a evolução dos sistemas operacionais ao longo do tempo, desde os sistemas de lote até os modernos sistemas com interfaces gráficas.
      \end{itemize}
      
\end{frame}

\backmatter
\end{document}
